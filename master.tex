\documentclass[a4paper,12pt]{article}

% General document formatting
%\usepackage[margin=0.7in]{geometry}
\usepackage[parfill]{parskip}
\usepackage{url, hyperref}
\usepackage{color}
\usepackage[usestackEOL]{stackengine}[2013-10-15] % formatting Pascal
\usepackage[dvipsnames]{xcolor}

\usepackage{cancel}
\usepackage[export]{adjustbox}

% Related to math
\usepackage{amsmath,amssymb,amsfonts,amsthm}
\usepackage{mathtools}
\usepackage{tikz}
\DeclarePairedDelimiter\norm{\lVert}{\rVert}
\newcommand{\innerproduct}[2]{\langle #1, #2 \rangle}

% encoding and language
\usepackage{lmodern}
\usepackage[slovene]{babel}
\usepackage[utf8]{inputenc}
\usepackage[T1]{fontenc}
\usepackage[mathscr]{euscript}

% multiline comments
\usepackage{verbatim}

% images
\usepackage{graphicx}
\graphicspath{ {./images/} }

% theorems
\theoremstyle{definition}
\newtheorem{counter}{Counter}[section] % not for use
\newtheorem{defn}[counter]{Definicija}
\newtheorem{lemma}[counter]{Lema}
\newtheorem{conseq}[counter]{Posledica}
\newtheorem{claim}[counter]{Trditev}
\newtheorem{theorem}[counter]{Izrek}
%%
\theoremstyle{remark}
\newtheorem*{ex}{Primer}
\newtheorem*{rem}{Opomba}
\newtheorem{rem*}[counter]{Opomba}
\newtheorem{ex*}[counter]{Primer}

% I like my squares DARK
\renewcommand\qedsymbol{$\blacksquare$}

% common commands redefined convenience purposes
\newcommand{\N}{\mathbb{N}}
\newcommand{\Z}{\mathbb{Z}}
\newcommand{\Q}{\mathbb{Q}}
\newcommand{\R}{\mathbb{R}}
\newcommand{\C}{\mathbb{C}}
\newcommand{\ch}{\operatorname{char}}


\begin{document}

\title{Numeri"cne metode 2\\ \small zapiski s predavanj prof. Marjetke Knez}
\author{Domen Vogrin}
\date{pomlad 2023}
\maketitle


\pagenumbering{roman}
\tableofcontents
\newpage
\pagenumbering{arabic}


% 13. 2. 2023
\section{Teorija aproksimacije}

\subsection{Aproksimacija funkcij}
Denimo, da imamo podano funkcijo $f$. Radi bi jo aproksimirali s kak"sno 'preprostej"so' funkcijo $\tilde{f}$, ki bi bila la"zje izra"cunljiva, bi se jo dalo enostavno odvajati, integrirati ...

\begin{ex}
    \[sin(x) \sim x - \frac{x^3}{3!} + \frac{x^5}{5!}\]
\end{ex}


Klju"cna vpra"sanja, ki se nam postavijo, so:
\begin{itemize}
    \item V kak"sni mno"zici/podprostoru naj i"s"cemo aproksimant $\tilde{f}$?
    \item V "cem naj si bo $\tilde{f}$ podobna/sorodna z $f$?
    \item Ali $\tilde{f}$ obstaja (v mno"zici, kjer jo i"s"cemo)?
    \item "ce obstaja, ali je dolo"cen enoli"cno?
    \item Kako konstruirati aproksimant $\tilde{f}$?
    \item Kako dobro nadomestilo za $f$ je izra"cunan $\tilde{f}$?
\end{itemize}

V splo"snem aproksimacijski problem formaliramo takole:

z $X$ ozna"cimo vektorski prostor, katerega elemente "zelimo aproksimirati, $S \subseteq X$ naj ozna"cuje podprostor/podmno"zico v $X$, v katerem i"s"cemo aproksimante. Aproksimacijska shema je operator 
\[\mathscr{A}\colon X \to S\]
ki vsakemu elementu $f \in X$ priredi aproksimacijski element (aproksimant) \[\tilde{f} = \mathscr{A} f \in S'\]

\begin{ex}
    Vektorski prostori:
    \begin{itemize}
        \item $X = \mathscr{C}([a, b]), X = \mathscr{C}^k([a, b])$
        \item $X = \mathscr{L}^{2}_{\rho}([a, b]) = \{f\colon[a, b] \to \R$ $\int_{a}^{b} \rho (x) dx < \infty \}$,\\
        pri "cemer je $\rho \textbf{ pozitivna ute"z: } \rho (x) > 0$ za vsak $x \in [a, b]$
        \item $X = \R ^n$
    \end{itemize}
\end{ex}

\begin{ex}
    Podprostori, v katerih i"s"cemo aproksimante:
    
    \begin{itemize}
        \item $S = P_n = Lin\{1, x, x^2, \dots, x^n\}$ polinom stopnje $\leq$ $n$ = \\
        $\{ \sum_{i = 0}^{n} a_i x^i; a_i \in \R \}$
        \item $S = Lin\{1, \sin x, \cos x, \sin 2x, \cos 2x, \dots, \sin nx, \cos nx\} \\
        \textbf{triginimetri"cni polinomi}$
        \item podprostori racionalnih funkcij, odsekoma polinomskih funkcij
    \end{itemize}
\end{ex}

Da bomo lahko definirali aproksimacijski problem in tudi ocenili napako aproksimacije, potrebujemo \textbf{normo}. Najbolj znane norme na prostoru funkcij so naslednje:

\begin{itemize}
    \item neskon"cna norma ($\norm{f}_{\infty}$)
    \[f \in \mathscr{C}([a, b]), \norm{f}_{\infty, [a, b]} = \max_{x \in [a, b]} |f(x)|\]
    Za izra"cun numeri"cnega pribli"zka za neskon"cno normo na intervalu $[a, b]$ izberemo dovolj gosto zaporedje to"ck:
    \[a \leq x_0 < x_1 < ... < x_n \leq b, \textbf{x} = (x_i)_{i=0}^N\]
    in izra"cunamo
    \[\norm{f}_{\infty, \textbf{x}} = \max_{i = 1, \dots, N} |f(x_i)|\]
    \item druga norma - norma, porojena iz skalarnega produkta
    Naj bo vektorski prostor $X$ opremljen s skalarnim produktom $\innerproduct{\cdot}{\cdot}$. Potem je $||f||_2 = \sqrt{\innerproduct{f}{f}}$.
    Primeri skalarnih produktov:
    \begin{enumerate}
        \item[$\cdot$] $\innerproduct{f}{g} = \int_{a}^{b} f(x) g(x) \rho(x) dx$, $f, g \in \mathscr{L}_{\rho}^2 ([a, b])$
        \item[$\cdot$] $\norm{f}_2 = \sqrt[]{\int_{a}^{b} f^2(x)\rho(x)dx }$
        
        Za $f(x) \equiv 1$ to imenujemo $\textbf{standardni skalarni produkt}$
    \end{enumerate}
    \item diskretni semi-skalarni produkt
    \[\textbf{x} = (x_i)_{i=0}^N, a \leq x_0 < x_1 < ... < x_n \leq b\]
    \[\innerproduct{f}{g} = \sum_{i = 0}^{N} f(x_i) g(x_i) \rho(x_i)\]
    "Ce ga "se delimo z dol"z"ino intervala, dobimo pribli"zek za prej"snjega.
    \[\norm{f}_{2, \textbf{x}} = \sqrt[]{\sum_{i = 0}^{N} f^2(x_i)\rho(x_i)}\]
\end{itemize}


Za dolo"canje aproksimanta $\tilde{f}$ lo"cimo dva primera:
\begin{enumerate}
    \item Optimalni aproksimacijski problemi
    \item interpolacija
\end{enumerate}

\subsubsection{Splo"sen optimalni aproksimacijski problem}
Naj bo $X$ vektorski prostor z normo $\norm{\cdot}$, $S \subseteq X$. Za $f \in X$ i"s"cemo $\tilde{f} \in S$, da velja
\begin{equation}
    \norm{f - \tilde{f}} = \inf_{s \in S} \norm{f - s} = dist(f, S)
\end{equation}
Torej, izmed mo"znih pribli"zkov izberemo najbolj"sega.


Pri tem predmetu si bomo ogledali:
\begin{enumerate}
    \item aproksimacijo po metodi najmanj"sih kvadratov
    
    (za normo izberemo drugo normo - normo iz skalarnega produkta)
    \item enakomerna polinomska aproksimacija ($X = C([a, b])$, $S = P_n$, $\norm{\cdot}_{\inf}$)
\end{enumerate}

Polinomi so zelo uporabni pri aproksimaciji funkcij, saj so gosti v prostoru zveznih funkcij.

\begin{theorem} (Weierstrassov izrek)
    Naj bo $f \in \mathscr{C} ([a, b])$. Potem za vsak $\varepsilon < 0$ obstaja polinom $p$, da je $\norm{f - p}_{\infty, [a, b]} < \varepsilon$. Drugače povedano:
    \begin{equation}
        dist(f, P_n) \to 0 \text{, ko gre } n \to \infty
    \end{equation}
\end{theorem}

\begin{proof}(konstruktivni - ideja)
    Naj bo $[a, b] = [0, 1]$. Za $f \in \mathscr{C} ([0, 1])$ definiramo t.i. $\textbf{Bernsteinov polinom}$:
    \begin{equation}
        \mathscr{B}_n f (x) = \sum_{i = 0}^{n} f (\frac{i}{n}) B_i^n(x)
    \end{equation}
    kjer je $B_i^n(x)$ $\textbf{Bernsteinov bazni polinom}$:
    \begin{equation}
        B_i^n (x) = {n \choose i} x^i (1-x)^{n-i} \text{, } i = 0, 1, \dots, n  
    \end{equation}
    Da se pokazati, da gre $\norm{f - \mathscr{B}_n f}_{\infty, [a, b]} \to 0$, ko gre $n \to \infty$.
\end{proof}

Bernsteinov aproksimacijski polinom nam poda en mo"zen na"cin aproksimacije funkcije $f$ (na $[0, 1]$).

Bernsteinov aproksimacijski operator:

\[\mathscr{B}_n : \mathscr{C} ([a, b]) \to P_n\]
\[f \mapsto \mathscr{B}_n f\]
\begin{equation}
    \mathscr{B}_n f(x) = \sum_{i = 0}^{n} f(a + \frac{i}{n}(b-a)) B_i^n (\frac{x-a}{b-a})
\end{equation}

Po Weierstrassovem izreku imamo zagotovljeno konvergenco v neskon"cni normi, "zal pa je konvergenca zelo po"casna.

\end{document}
