\documentclass{article}
\usepackage{graphicx} % Required for inserting images

\title{NM2}
\author{Domen Vogrin}
\date{pomlad 2023}

\begin{document}

\maketitle

% 13. 2. 2023
\section{Teorija aproksimacije}

\subsection{Aproksimacija funkcij}
Denimo, da imamo podano funkcijo $f$. Radi bi jo aproksimirali s kakšno "preprostejšo" funkcijo $\tilde{f}$, ki bi bila lažje izračunljiva, bi se jo dalo enostavno odvajati, integrirati ...

primer:
\begin{equation}
    sin(x) ~ x - \frac{x^3}{3!} + \frac{x^5}{5!}
\end{equation}

\subsection{Ključna vprašanja}
\begin{itemize}
    \item V kakšni množici/podprostoru naj iščemo aproksimant $\tilde{f}$?
    \item V čem naj si bo $\tilde{f}$ podobna/sorodna z $f$?
    \item Ali $\tilde{f}$ obstaja (v množici, kjer jo iščemo)?
    \item Če obstaja, ali je določen enolično?
    \item Kako konstruirati aproksimant $\tilde{f}$?
    \item Kako dobro nadomestilo za $f$ je izračunan $\tilde{f}$?
\end{itemize}

V splošnem aproksimacijski problem formaliramo takole:

z $X$ označimo vektorski prostor, katerega elemente želimo aproksimirati, $S 	\subseteq X$ naj označuje podprostor/podmnožico v $X$, v katerem iščemo aproksimante. Aproksimacijska shema je operator $A: X \longrightarrow S$, ki vsakemu elementu $f \in X$ priredi aproksimacijski element (aproksimant) $\tilde{f} = A f \in S'$

Primeri vektorskih prostorov:
\begin{itemize}
    \item $X = C([a, b]), X = C^k([a, b])$
    \item $X = L^{2}_{\rho}([a, b]) = \{f: [a, b] \longrightarrow \R (int1)\}$, pri čemer je $\rho$ POZITIVNA UTEŽ: (pu1)
    \item $X = \R ^n$
\end{itemize}

Primeri podprostorov, v katerih iščemo aproksimante:
\begin{itemize}
    \item $S = \P_n = Lin\{1, x, x^2, ..., x^n\}$ polinom stopnje $\leq$ $n$ = #3
\end{itemize}

Da bomo lahko definirali aproksimacijski problem in tudi ocenili napako aproksimacije, potrebujemo \textbf{normo}.

Najbolj znane norme na prostoru funkcij so naslednje:
\begin{itemize}
    \item neskončna norma
    \item druga norma - norma, porojena iz skalarnega produkta
    Naj bo vektorksi prostor $X$ opremljen s skalarnim produktom $<., .>$. Potem je $||f||_2 = \sqrt{<f, f>}$.
    Primeri skalarnih produktov (glej list)
    \item diskretni semi-skalarni produkt
\end{itemize}

Kako določiti aproksimant f~?

Ločimo:
\begin{enumerate}
    \item Optimalni aproksimacijski problemi
    \item interpolacija
\end{enumerate}

Splošen optimalni aproksimacijski problem:
Naj bo $X$ vektorski prostor z normo ||.|| $S \subseteq X$. Za $f \in X$ poiščemo $f~ \in S$, da velja #5

Pri tem predmetu si bomo ogledali:
\begin{enumerate}
    \item aproksimacijo po metodi najmanjših kvadratov
    (za normo izberemo drugo normo - normo iz skalarnega produkta)
    \item enakomerna polinomska Aproksimacija
    $X = C([a, b]), S = \P_n, ||.||_{\inf}$ 
\end{enumerate}

Polinomi so zelo uporabni pri aproksimaciji funkcij, saj so gosti v prostoru zveznih funkcij:

Izrek: Weierstrassov izrek

Bernsteinov aproksimacijski polinom nam poda en možen način aproksimacije funkcije $f$ (na $[0, 1]$).

Bernsteinov aproksimacijski operator:

Po Weierstrassovem izreku imamo zagotovljeno konvergenco v neskončni normi, žal pa je konvergenca zelo počasna.

\end{document}
